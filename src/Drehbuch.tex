\documentclass{screenplay}

\title{FNMA-Werbevideo für den GPR und das mobile Aufnahmestudio}
\author{TOP Lehre}
\realauthor{Talha Dursun}
\address{
	Muldenstraße 54\\
	4020 Linz
}

\begin{document}
	\coverpage
	\fadein
	
	\intslug[tag]{studentenwohnung - eingangshalle}
	
	Die Hintergrundmusik beginnt zu spielen. POV - das Hauptkamerabild des Videos - biegt um die Ecke und tritt in den GPR hinein.
	
	\intslug[tag]{studentenwohnung - gpr}
	
	Die Vorhänge sind zugezogen. Der Raum wird von den Elgato-Führungslichtern beleuchtet. POV schwenkt langsam im Raum von links nach rechts.
	
	\begin{dialogue}[nüchterner Ton]{sprecher (v.o.)}
		Der General Production Room ist ein Aufnahmestudio, in dem Lehrende sowie Studierende schnell und ohne großen Aufwand Videos aufnehmen können.
	\end{dialogue}
	
	Die Kameras und Mikrofone für die Interviewsituation sind in Betrieb. PERSON \#1 und \#2 sitzen auf zwei Stühlen und führen ein Gespräch. Sie sind links in unscharfem Hintergrund. KAMERA \#2 steht rechts in scharfem Vordergrund und filmt.
	
	\begin{dialogue}{sprecher}
		Mithilfe der drei Kameras und Mikrofone kann man entweder eine Einzelaufnahme oder ein Interview filmen.
	\end{dialogue}
	
	Der Vordergrund verschwimmt, während Person \#1 und \#2 scharfgestellt werden.
	
	POV wechselt auf einen LEHRER, der gerade eine Einzelaufnahme filmt.
	
	\begin{dialogue}{sprecher}
		Für die Einzelaufnahme gibt es einen Projektor, Fernseher, Greenscreen und einen Stream Deck Kontroller\ldots
	\end{dialogue}
	
	\begin{dialogue}{sprecher}
		Dementsprechend ist der Raum mit Mobiliar, schalldichten Vorhängen sowie Polstern und vielerlei Hilfsmitteln ausgestattet.
	\end{dialogue}
	
	\begin{dialogue}{sprecher}
		Das Herzstück des Studios sind der Schnittrechner und der ATEM Mischer. Der ATEM erlaubt es, mehrere Eingangssignale für einen professionellen Auftritt zu mischen. Am Schnittrechner kann man die fertige Aufzeichnung überprüfen und nachbearbeiten.
	\end{dialogue}
	
	[Streamdeck und seine Funktionen erklären.]
	
	[übergang von GPR zu koffer: "und wenn der raum für euch zu klein ist, haben wir eine kleinere variante" oder so ähnlich.]
	
	[fokuswechsel von interviewszene auf kamerabildschirm: https://youtu.be/r-6eXC8VDFI?si=ODd1on14AZmXkKb1\&t=72]
	
	[kameraschwenk von interview zum mischpult á la https://youtu.be/3RwhtpAhv1Q?feature=shared\&t=47]
	
	[produktshot von koffer im HYPE raum: https://youtu.be/I9qHchu2m7Y?si=rnbpYHzzRIep9I5n]
	
	[zwei schauspieler machen ein interview während ein dritter hinter der kamera / am mischpult sitzt. 3rd person shot, wo alle drei sichtbar sind.]
	
	[shot von einzelaufnahme szenario aus 3rd person, dann übergang zur tatsächlichen aufnahme]
	
	[zeitraffer, wie jemand den koffer im besprechungsraum aufsetzt, dann shot in normalem tempo, wie er sich mit der kamera filmt und unterricht gibt]
	
	[shot von draußen, wie jemand den koffer ins fh3 gebäude rollt]
	
	[evtl. drohnenshot vom campus]
	
	[aufnahme von gpr in betrieb, mit mehreren eingangsquellen beim atem zb laptop]
	
	[erwähnen, dass nach einmaliger einschulung gpr ohne betreuung nutzbar ist.]
	
	[gendern und geschlechtergerechter sprechertext]
	
	[smoother übergang: https://youtu.be/EcNVHzS86Hc?feature=shared\&t=45]
	
	[beispiel von kufstein, wie sie ihren schnittplatz vorstellt: https://youtu.be/EcNVHzS86Hc?feature=shared\&t=138]
	
	\fadeout
	\theend
\end{document}