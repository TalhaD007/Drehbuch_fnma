\documentclass{screenplay}

\title{
	FNMA-Werbevideo für den GPR\\
	und das mobile Aufnahmestudio
}
\author{TOP Lehre}
\realauthor{Talha Dursun}
\address{
	Muldenstraße 54\\
	4020 Linz
}

\begin{document}
	\coverpage
	\fadein
	
	\intslug[tag]{studentenwohnung - eingangshalle}
	
	Die Hintergrundmusik beginnt zu spielen. POV (die Kameraperspektive im Video) biegt um die Ecke und tritt in den GPR ein.
	
	\intslug[tag]{studentenwohnung - gpr}
	
	Die Vorhänge sind aufgezogen. POV schwenkt langsam im Raum von links nach rechts. SPRECHER beginnt zu reden.
	
	\begin{dialogue}[nüchtern]{sprecher (v.o.)}
		Der General Production Room ist ein Aufnahmestudio, in dem Lehrende sowie Studierende ohne großen Aufwand Videos erstellen können. Dank seiner schalldichten Dämmungen und Vorhänge bietet er eine optimale Raumakustik und Ambiente.
	\end{dialogue}
	
	PERSON \#1 zieht die Vorhänge zu und schaltet die Stromverteiler ein (Filmmontage).
	
	Die gesamte Filmausstattung leuchtet auf.
	
	PERSON \#2 und \#3 sitzen auf zwei Stühlen und führen ein Gespräch. Kamera 1 ist auf die beiden gerichtet. Im POV sind die Personen im scharfen Hintergrund und Kamera 1 im unscharfen Vordergrund zu sehen.
	
	\begin{dialogue}{sprecher}
		Zudem ist der Raum mit drei Kameras und Mikrofonen ausgestattet,\dots
	\end{dialogue}
	
	Der Hintergrund verschwimmt, während der Vordergrund scharfgestellt wird.
	
	\begin{dialogue}{sprecher (cont'd)}
		\dots die hochwertiges Filmmaterial aufnehmen und dieses zum Mischpult übertragen.
	\end{dialogue}
	
	POV schwenkt von Kamera \#2 auf Person \#1 um. Sie bedient den Mischer und ändert das Anzeigebild.
	
	\begin{dialogue}{sprecher}
		Der Mischer bearbeitet das laufende Filmmaterial mit zusätzlichen Medien und Bildern. Das Ergebnis wird auf der Festplatte gespeichert und bei Bedarf am Schnittrechner nachbearbeitet.
	\end{dialogue}
	
	Person \#1 steckt die Festplatte aus, schließt sie an das Laptop an und schneidet das Video in einem Schnittprogramm.
	
	PERSON \#4 filmt sich selbst am Stehtisch. POV schwenkt vom Tisch zum Mischpult um. Der Bildschirm ist nach vorne gerichtet, sodass Person \#4 sich selbst mit ihren PowerPoint-Folien (Bild-im-Bild) sehen kann.
	
	\begin{dialogue}{sprecher}
		Der GPR eignet sich sogar für den Einzelbetrieb. Denn mithilfe des Stream Deck Controllers lässt sich das Mischpult fernsteuern. Die Benutzeroberfläche ist sowohl auf geübte als auch unerfahrene Nutzer und Nutzerinnen zugeschnitten.
	\end{dialogue}
	
	POV blickt auf den Controller, während Person \#4 die Ansicht von "Mixer" auf "Mixer (Simple)" wechselt und auf Start drückt.
	
	ANTON und Person \#4 sitzen am Schreibtisch. Anton zeigt vor, wie der Mischer bedient wird.
	
	\begin{dialogue}{sprecher}
		Dasselbe gilt für das gesamte Studio. Nach einer einmaligen Schulung kann nämlich jeder den Raum ohne weitere Betreuung benutzen.
	\end{dialogue}
	
	\dots
	
	\begin{dialogue}{sprecher}
		Sollten dennoch Probleme auftauchen, steht das TOP Lehre Team jederzeit zur Verfügung.
	\end{dialogue}
	
	\extslug[tag]{auf dem weg zum fh3-gebäude}
	
	POV schaut zum Eingang aus der Froschperspektive auf.
	
	\begin{dialogue}[ironisch]{sprecher}
		Falls das Studio doch zu eng wird, steht auch die mobile Variante zur Auswahl.
	\end{dialogue}
	
	Person \#1 kommt mit dem Aufnahmekoffer ins Bild und geht in Richtung FH3-Gebäude.
	
	\intslug{fotostudio}
	
	Der Koffer sitzt offen auf einer weißen Plane mit schwarzem Hintergrund. POV geht langsam um den Koffer herum.
	
	\begin{dialogue}[wieder nüchtern]{sprecher}
		Der Aufnahmekoffer wurde entwickelt, um die Vorteile eines Mischpults genießen zu können, ohne an einem Standort gebunden zu sein.
	\end{dialogue}
	
	\intslug[tag]{besprechungsraum}
	
	Person \#1 stellt den Koffer im Zeitraffer auf. Danach hält sie einen Unterricht vor der Tafel.
	
	\begin{dialogue}{sprecher}
		Der Koffer ist leicht zum Aufsetzen und dank seines ergonomischen Designs benutzerfreundlich.
	\end{dialogue}
	
	POV zeigt den Koffer in Betrieb, während Person \#1 weiter filmt.
	
	\dots
	
	[evtl. drohnenshot vom campus]
	
	\fadeout
	\theend
\end{document}