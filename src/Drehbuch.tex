\documentclass{screenplay}

\title{Werbevideo FNMA}
\author{TOP Lehre}
\realauthor{Talha Dursun}
\address{
	Muldenstraße 54\\
	4020 Linz
%	Softwarepark 11\\
%	4232 Hagenberg
}

\begin{document}
	\coverpage
	\fadein
	
	\intslug[tag]{GPR, Eingangshalle.}
	
	POV geht im Zeitraffer in Richtung Raumplan von GPR. Er biegt um die Ecke und geht langsam in den GPR hinein.
	
	\begin{dialogue}{SPRECHER}
		Der General Production Room ist ein Aufnahmestudio, in dem Lehrende sowie Studierende schnell und unkompliziert Videos aufnehmen können.
	\end{dialogue}
	
	\intslug[tag]{GPR.}
	
	\begin{dialogue}{SPRECHER}
		Mithilfe der drei Kameras und Mikrofone kann man entweder eine Einzelaufnahme oder ein Interview filmen.
	\end{dialogue}
	
	\begin{dialogue}{SPRECHER}
		Dementsprechend ist der Raum mit Mobiliar, schalldichten Vorhängen sowie Polstern und vielerlei Hilfsmitteln ausgestattet.
	\end{dialogue}
	
	[Streamdeck und seine Funktionen erklären.]
	
	\begin{dialogue}{SPRECHER}
		Das Herzstück des Studios sind der Schnittrechner und der ATEM Mischer. Der ATEM erlaubt es, mehrere Eingangssignale für einen professionellen Auftritt zu mischen. Am Schnittrechner kann man die fertige Aufzeichnung überprüfen und nachbearbeiten.
	\end{dialogue}
	
	\fadeout
	\theend
\end{document}
