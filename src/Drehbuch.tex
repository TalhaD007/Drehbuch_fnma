\documentclass{screenplay}

\title{
	FNMA-Werbevideo für den GPR\\
	und das mobile Aufnahmestudio
}
\author{TOP Lehre}
\realauthor{Talha Dursun}
\address{
	Muldenstraße 54\\
	4020 Linz
}

\begin{document}
	\coverpage
	\fadein
	
	\intslug[tag]{studentenwohnung - vorraum}
	
	Die Musik beginnt zu spielen. POV (Kameraperspektive des Videos) biegt um die Ecke und tritt in den GPR ein.
	
	\intslug[tag]{studentenwohnung - gpr}
	
	Der GPR ist in Betrieb. POV schwenkt langsam im Raum von links nach rechts.
	
	\begin{dialogue}[nüchtern]{sprecher (v.o.)}
		Der General Production Room der FH Oberösterreich am Campus Hagenberg ist ein Aufnahmestudio, in dem Lehrende sowie Studierende ohne großen Aufwand Videos erstellen können.
	\end{dialogue}
	
	PERSON \#1 hängt die Schalldämmung auf und zieht die Vorhänge zu.
	
	\begin{dialogue}{sprecher}
		Dank seiner schalldichten Dämmungen und Vorhänge bietet der Raum eine optimale Akustik und Ambiente.
	\end{dialogue}
	
	Person \#1 schaltet die Stromverteiler ein. Die gesamte Filmausstattung leuchtet auf (Filmmontage).
	
	PERSON \#2 und \#3 sitzen auf zwei Stühlen im Hintergrund und führen ein Gespräch. Kamera 1 ist auf die beiden gerichtet und unscharf im Vordergrund zu sehen.
	
	\begin{dialogue}{sprecher}
		Er ist mit drei Kameras, Mikrofonen und sechs LED-Lampen ausgestattet \dots
	\end{dialogue}
	
	Der Hintergrund verschwimmt, während der Vordergrund scharfgestellt wird.
	
	\begin{dialogue}{sprecher (cont'd)}
		\dots~die hochwertiges Filmmaterial aufnehmen und dieses zum Mischpult übertragen.
	\end{dialogue}
	
	Person \#1 bedient den Mischer und ändert das Anzeigebild von Person \#2 auf \#3.
	
	\begin{dialogue}{sprecher}
		Der Mischer bearbeitet das laufende Filmmaterial mit zusätzlichen Medien und Bildern. Alle Audio- und Videoquellen sowie das gemischte Video werden getrennt auf der Festplatte gespeichert. Das Ergebnis kann man direkt am Schnittrechner nachbearbeiten.
	\end{dialogue}
	
	Person \#1 steckt die Festplatte aus und schließt sie am Laptop an (Nahaufnahme).
	
	Person \#1 schneidet das Video in einem Schnittprogramm (Über-die-Schulter).
	
	\begin{dialogue}{sprecher}
		Der GPR eignet sich für Interviews und Einzelaufnahmen. Der Stream Deck Controller steuert das Mischpult aus der Ferne. Die Benutzeroberfläche ist sowohl für geübte als auch unerfahrene Nutzer und Nutzerinnen geeignet.
	\end{dialogue}
	
	PERSON \#4 filmt ein Lernvideo am Stehtisch.
	
	Der Bildschirm des Mischers ist nach vorne gerichtet, sodass Person \#4 sich selbst mit ihren digitalen Lehrinhalten sehen kann.
	
	Person \#4 wechselt die Ansicht von "Mixer" auf "Mixer (Simple)" und drückt auf Start (Nahaufnahme).
	
	\begin{dialogue}{sprecher}
		Dasselbe gilt für das gesamte Studio. Nach einer einmaligen Schulung kann jeder den Raum ohne weitere Betreuung benutzen. Sollten dennoch Probleme auftauchen, steht das TOP Lehre Team jederzeit zur Verfügung.\paren{ironisch}Falls das Studio aber zu eng wird, steht auch die mobile Variante zur Auswahl.
	\end{dialogue}
	
	ANTON und Person \#4 sitzen am Schreibtisch. Anton zeigt vor, wie der Mischer bedient wird.
	
	POV geht langsam von links nach rechts und behält den Blick auf die beiden. POV geht weiter, bis die Eingangswand den Blick zu Anton verdeckt.
	
	\extslug[tag]{fh3-gebäude}
	
	POV bewegt sich weiter nach rechts und erblickt das FH3-Gebäude. Person \#1 kommt von links ins Bild und rollt den Aufnahmekoffer mit sich.
	
	\begin{dialogue}{sprecher}
		Der Aufnahmekoffer wurde entwickelt, um semi-professionelles Filmequipment nutzen zu können, ohne an einem Standort gebunden zu sein.
	\end{dialogue}
	
	\intslug{fotostudio}
	
	Der Koffer sitzt offen auf einer weißen Plane mit schwarzem Hintergrund. POV dreht sich langsam um den Koffer herum.
	
	\begin{dialogue}{sprecher}
		Der Koffer ist leicht zum Aufbauen und dank seines ergonomischen Designs benutzerfreundlich.
	\end{dialogue}
	
	\intslug[tag]{besprechungsraum}
	
	Person \#1 stellt den Koffer im Zeitraffer auf. Daraufhin hält sie einen Unterricht vor einer Pinnwand mit Notizen.
	
	Der Koffer ist in Betrieb. Person \#1 ist auf dem Monitor zu sehen (Nahaufnahme).
	
	\extslug[tag]{campus Hagenberg - Wald}
	
	POV schaut aus der Vogelperspektive auf den Campus herab und fliegt langsam nach links.
	
	\begin{dialogue}{sprecher}
		Das TOP Lehre Team am Campus Hagenberg stellt den GPR und Aufnahmekoffer jedem zur Verfügung. Sein Ziel ist es, die Qualität der Lehre mit Services wie diesen zu verbessern. Wir freuen uns auf die vielen, spannenden Projekte, die daraus entstehen werden.
	\end{dialogue}
	
	\extslug[tag]{studentenwohnung - eingang}
	
	Die Musik klingt ab. Das Schild des GPR hängt mit der Rückseite "RECORDING IN PROGRESS" an der Tür. Die Eingangstür wird zugesperrt.
	
	\fadeout
	\theend
\end{document}