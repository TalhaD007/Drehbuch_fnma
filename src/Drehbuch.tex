\documentclass{screenplay}

\title{
	FNMA-Werbevideo für den GPR\\
	und das mobile Aufnahmestudio
}
\author{TOP Lehre}
\realauthor{Talha Dursun}
\address{
	Muldenstraße 54\\
	4020 Linz
}

\begin{document}
	\coverpage
	\fadein
	
	\extslug[tag]{campus hagenberg - vorhof}
	
	Vorspann wird abgespielt mit Jingle. Sobald die Titelkarte ausblendet, beginnt die Hintergrundmusik zu spielen.
	
	\intslug[tag]{studentenwohnung - vorraum}
	
	POV (Kameraperspektive des Videos) betritt den GPR, biegt um die Ecke und zieht den Vorhang auf.
	
	\intslug[tag]{studentenwohnung - gpr}
	
	Der GPR ist in Betrieb. POV schwenkt langsam im Raum von links nach rechts und schaut sich die Filmausrüstung an.
	
	\begin{dialogue}[nüchtern]{sprecher (v.o.)}
		Der General Production Room der FH Oberösterreich am Campus Hagenberg ist ein Aufnahmestudio, in dem Lehrende sowie Studierende ohne großen Aufwand Videos erstellen können.
	\end{dialogue}
	
	PERSON \#1 hängt die Schalldämmung auf und zieht die Vorhänge zu.
	
	\begin{dialogue}{sprecher}
		Dank seiner schalldichten Dämmungen und Vorhänge bietet der Raum eine optimale Akustik und Ambiente.
	\end{dialogue}
	
	Nahaufnahmen von jeweils einer Kamera, einem Mikrofon und einer Leuchte.
	
	\begin{dialogue}{sprecher}
		Er ist mit drei Kameras, Mikrofonen und sechs LED-Lampen ausgestattet \dots
	\end{dialogue}
	
	PERSON \#2 und \#3 sitzen auf zwei Stühlen im Hintergrund und führen ein Gespräch.
	
	Kamera 1 ist auf die beiden gerichtet und im Vordergrund zu sehen.
	
	\begin{dialogue}{sprecher (cont'd)}
		\dots~die hochwertiges Filmmaterial aufnehmen und dieses zum Mischpult übertragen.
	\end{dialogue}
	
	Person \#1 bedient den Mischer und ändert das Anzeigebild von Person \#2 auf \#3.
	
	\begin{dialogue}{sprecher}
		Der Mischer bearbeitet das laufende Filmmaterial mit zusätzlichen Medien und Bildern. Alle Audio- und Videoquellen sowie das gemischte Video werden getrennt auf der Festplatte gespeichert. Das Ergebnis kann man direkt am Schnittrechner nachbearbeiten.
	\end{dialogue}
	
	Person \#1 steckt die Festplatte aus und schließt sie am Laptop an (Nahaufnahme).
	
	Person \#1 schneidet das Video in einem Schnittprogramm (Über-die-Schulter).
	
	\begin{dialogue}{sprecher}
		Der GPR eignet sich für Interviews und Einzelaufnahmen. Der Stream Deck Controller steuert das Mischpult aus der Ferne. Die Benutzeroberfläche ist sowohl für geübte als auch unerfahrene Nutzer und Nutzerinnen geeignet.
	\end{dialogue}
	
	PERSON \#4 filmt ein Lernvideo am Stehtisch.
	
	Der Bildschirm des Mischers ist nach vorne gerichtet, sodass Person \#4 sich selbst mit ihren digitalen Lehrinhalten sehen kann.
	
	Person \#4 wechselt die Ansicht von "Mixer" auf "Mixer (Simple)" und drückt auf Start (Nahaufnahme).
	
	\begin{dialogue}{sprecher}
		Dasselbe gilt für das gesamte Studio. Nach einer einmaligen Schulung kann jeder den Raum ohne weitere Betreuung benutzen. Sollten dennoch Probleme auftauchen, steht das TOP Lehre Team jederzeit zur Verfügung.\paren{ironisch}Falls das Studio aber zu eng wird, steht auch die mobile Variante zur Auswahl.
	\end{dialogue}
	
	Person \#4 und \#1 sitzen am Schreibtisch. Person \#1 zeigt vor, wie der Mischer bedient wird.
	
	JULIA und ANTON erscheinen im Bild, sobald die TOP Lehre erwähnt wird.
	
	POV geht langsam von links nach rechts und behält den Blick auf die beiden. POV geht weiter, bis der Vorhang das Bild verdeckt.
	
	\extslug[tag]{fh3-gebäude}
	
	POV bewegt sich weiter nach rechts und erblickt das FH3-Gebäude. Person \#1 kommt von links ins Bild und rollt den Aufnahmekoffer mit sich.
	
	\begin{dialogue}{sprecher}
		Der Aufnahmekoffer wurde entwickelt, um semi-professionelles Filmequipment nutzen zu können, ohne an einem Standort gebunden zu sein. Der Koffer ist leicht zum Aufbauen und dank seines ergonomischen Designs benutzerfreundlich.
	\end{dialogue}
	
	\intslug[tag]{besprechungsraum}
	
	Person \#1 stellt den Koffer in einer Montage auf.
	
	Der Koffer ist in Betrieb. Person \#1 bereitet die Aufnahme vor (Über-die-Schulter).
	
	\extslug[tag]{campus hagenberg - vorhof}
	
	POV schaut aus der Vogelperspektive auf den Campus herab und fliegt langsam nach links.
	
	\begin{dialogue}{sprecher}
		Das TOP Lehre Team am Campus Hagenberg stellt den GPR und Aufnahmekoffer jedem zur Verfügung. Sein Ziel ist es, die Qualität der Lehre mit Services wie diesen zu verbessern. Wir freuen uns auf die vielen, spannenden Projekte, die daraus entstehen werden.
	\end{dialogue}
	
	\extslug[tag]{studentenwohnung - eingang}
	
	Die Musik klingt ab. Das Schild des GPR hängt mit der Rückseite "RECORDING IN PROGRESS" an der Tür. Die Eingangstür wird zugesperrt.
	
	\fadeout
	\theend
\end{document}