\documentclass{screenplay}

\begin{document}
	Der General Production Room der FH Oberösterreich am Campus Hagenberg ist ein Aufnahmestudio, in dem Lehrende sowie Studierende ohne großen Aufwand Videos erstellen können.
	
	Dank seiner schalldichten Dämmungen und Vorhänge bietet der Raum eine optimale Akustik und Ambiente.
		
	Er ist mit drei Kameras, Mikrofonen und sechs LED-Lampen ausgestattet, die hochwertiges Filmmaterial aufnehmen und dieses zum Mischpult übertragen.
	
	Der Mischer bearbeitet das laufende Filmmaterial mit zusätzlichen Medien und Bildern.
	
	Alle Audio- und Videoquellen sowie das gemischte Video werden getrennt auf der Festplatte gespeichert.
	
	Das Ergebnis kann man direkt am Schnittrechner nachbearbeiten.
	
	Der GPR eignet sich für Interviews und Einzelaufnahmen.
	
	Der Stream Deck Controller steuert das Mischpult aus der Ferne.
	
	Die Benutzeroberfläche ist sowohl für geübte als auch unerfahrene Nutzer und Nutzerinnen geeignet.
		
	Dasselbe gilt für das gesamte Studio.
	
	Nach einer einmaligen Schulung kann jeder den Raum ohne weitere Betreuung benutzen.
	
	Sollten dennoch Probleme auftauchen, steht das TOP Lehre Team jederzeit zur Verfügung.
	
	Falls das Studio aber zu eng wird, steht auch die mobile Variante zur Auswahl.
		
	Der Aufnahmekoffer wurde entwickelt, um semi-professionelles Filmequipment nutzen zu können, ohne an einem Standort gebunden zu sein.
		
	Der Koffer ist leicht zum Aufbauen und dank seines ergonomischen Designs benutzerfreundlich.
		
	Das TOP Lehre Team am Campus Hagenberg stellt den GPR und Aufnahmekoffer jedem zur Verfügung.
	
	Sein Ziel ist es, die Qualität der Lehre mit Services wie diesen zu verbessern.
	
	Wir freuen uns auf die vielen, spannenden Projekte, die daraus entstehen werden.
\end{document}